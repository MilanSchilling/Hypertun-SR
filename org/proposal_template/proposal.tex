% ETH Zurich  - 3D Vision 2016
% http://www.cvg.ethz.ch/teaching/3dvision/
% Template for project proposals

\documentclass[11pt,a4paper,oneside,onecolumn]{IEEEtran}
\usepackage{graphicx}

% Enter the project title and your project supervisor here
\newcommand{\ProjectTitle}{High-Performance and Tunable Stereo Reconstruction}
\newcommand{\ProjectSupervisor}{Peidong Liu}
\newcommand{\DateOfReport}{March 10, 2017}

% Enter the team members' names and path to their photos. Comment / uncomment related definitions if the number of members are different than 2.
% Including photographs is optional but highly encouraged. Photos are there to help us to evaluate your group more effectively. 
% If you wish not to include your photos, please comment out the following line.
\newcommand{\PutPhotos}{}
% Please include a clear photo of each member. (use pdf or png files for Latex to embed them in the document well)
\newcommand{\memberone}{Ian Staehli}
\newcommand{\memberonepicture}{ian.jpg}
\newcommand{\membertwo}{Johann Diep}
\newcommand{\membertwopicture}{johann.jpg}
\newcommand{\memberthree}{Milan Schilling}
\newcommand{\memberthreepicture}{milan.jpg}
%\newcommand{\memberfour}{Member Name}
%\newcommand{\memberfourpicture}{pic4.png}


%%%% DO NOT EDIT THE PART BELOW %%%%
\title{\ProjectTitle}
\author{3D Vision Project Proposal\\Supervised by: \ProjectSupervisor\\ \DateOfReport}
\begin{document}
\maketitle
\vspace{-1.5cm}\section*{Group Members}\vspace{0.3cm}
\begin{center}\begin{minipage}{\linewidth}\begin{center}
\begin{minipage}{3 cm}\begin{center}\memberone\ifdefined\PutPhotos\\\vspace{0.2cm}\includegraphics[height=3cm]{\memberonepicture}\fi\end{center}\end{minipage}
\ifdefined\membertwo\begin{minipage}{3 cm}\begin{center}\membertwo\ifdefined\PutPhotos\\\vspace{0.2cm}\includegraphics[height=3cm]{\membertwopicture}\fi\end{center}\end{minipage}\fi
\ifdefined\memberthree\begin{minipage}{3 cm}\begin{center}\memberthree\ifdefined\PutPhotos\\\vspace{0.2cm}\includegraphics[height=3cm]{\memberthreepicture}\fi\end{center}\end{minipage}\fi
\ifdefined\memberfour\begin{minipage}{3 cm}\begin{center}\memberfour\ifdefined\PutPhotos\\\vspace{0.2cm}\includegraphics[height=3cm]{\memberfourpicture}\fi\end{center}\end{minipage}\fi
\end{center}\end{minipage}\end{center}
%%%% END OF PROTECTED LINES %%%%


%%%% BEGIN WRITING THE DOCUMENT HERE %%%%

\section{Description of the project}

Conventional stereo algorithms are focused on getting qualitative reconstruction on datasets without considering run time performance, which results in the employment of computationally expensive techniques. Many applications such as mobile robots require fast perception of their surrounding in order to move and perform tasks in real-time. Therefore, this project is concerned with the implementation of a high-performance stereo disparity estimation algorithm described in \cite{paper}. It approximates large-scale disparities with a planar mesh. It is placed with sparsely matched keypoints at the beginning, and gets refined with every iteration. Hence it is possible to adjust the accuracy-versus-speed trade-off to the practical requirements.

\section{Work packages and timeline}

Detailed descriptions of work packages you planned, their outcomes, the responsible group member and estimated timeline. Specify the challenges that will be tackled and considered solutions with possible alternatives, citing related documents if applicable. Mention the platform (Android, PC etc.) and the language (C++ etc.) you plan to use.

\section{Outcomes and Demonstration}

Give detailed information on the expected outcome of your project and the experiments you plan to test your implementation. If applicable, describe the online or offline demo you plan to present at the end of the semester.


\vspace{1cm}
\textbf{Instructions:}

\begin{itemize}
\item The document should not exceed two pages including the references.
\item Please name the document
  \textbf{3DVision\_Proposal\_Group\_\emph{\#}.pdf} and send it to
  Federico Camposeco in an email titled \textbf{[3D Vision] Project Proposal - Group \#}, filling in your group number.
\end{itemize}

{%\singlespace
{\small
\bibliography{refs}
\bibliographystyle{plain}}}




\end{document}
